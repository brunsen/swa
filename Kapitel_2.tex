\chapter{DevOps}
Unter dem Begriff DevOps werden eine Menge an Maßnahmen, die die Bereiche Development und Operations enger zusammenführen, verstanden. In diesem Kapitel wird zunächst die Motivation von DevOps beschrieben. Im Anschluss werden einige Maßnahmen, die DevOps ermöglichen vorgestellt. Zum Abschluss des Kapitels werden noch ein mal die Ziele von DevOps hervorgehoben.

\section{Motivation}
DevOps bietet einem Softwarehersteller viele Vorteile. Unter anderem wird das  Prinzip des Continuous Deployment, das die Zeit zwischen der Entwicklung einer Anwendung und dem Release dieser Anwendung verringert, ermöglicht. Beim Continuous Deployment wird direkt nachdem ein Entwicklerteam neuen Code innerhalb einer Versionsverwaltung wie zum Beispiel Git freigegeben hat, dieser mittels Automatisierung auf die Zielsysteme gespielt. Dadurch ermöglicht Continuous Deployment einem Softwarehersteller neue Features schneller zu veröffentlichen und unter Umständen einen Vorteil gegenüber der Konkurrenz zu erzielen.\\
Zustätzlich sollte die benötigte Koordination zwischen Development und Operations verringert werden. Des weiteren ermöglicht DevOps einsparungen in den benötigten Kapazitäten des Operation Teams. Dies wird mittels Automatisierung gängiger Aufgaben von Operations wie der Analyse von Systemlogs, der Erstellung von Backups, der Optimierung von Systemperformance oder der Lösung von identifizierten Problemen ermöglicht.

\section{Maßnahmen}
Wie bereits erwähnt umfasst der Begriff DevOps die Maßnahmen, die Development und Operations näher zusammenführen sollen. Eine dieser Maßnahmen ist die Berücksichtigung von Operations bei der Softwareentwicklung. Da Operations auf Werkzeuge zum Logging und Monitoring einer Anwendung angewiesen sind, sollen Entwickler geignete Logs und Schnittstellen zum Monitoring innerhalb der Anwendung zur Verfügung stellen. Zusätzlich sollen die Entwickler eine möglichst hohe Anzahl an Informationen für das Incident Handling bereit stellen. Dies kann durch eine hohe Anzahl an beschriebenen Fehlertypen innerhalb der Anwendung realisiert werden. \\
Zusätzlich fordert DevOps, dass Entwickler für das Behandeln ernsterer Incidents (Vorfälle) verantwortlich gemacht werden. Sollte ein Programm einen Fehler aufweisen, der zum Beispiel nicht trivial durch eine angepasste Umgebungsvariable zu lösen ist, sollen sich die Entwickler des Problems annehmen. \\
Des weiteren erfordert DevOps, dass Development und Operations denselben Deployment Prozess verwenden. Das bedeutet, dass Entwickler während der Entwicklung ihren Programmcode mit den selben Werkzeugen beziehungsweise Skripten auf ihre Testumgebung spielen, die Operation  in der Transitionphase ebenfalls verwenden würden.\\
Zusätzlich legt DevOps großen Wert eine saubere Infrastruktur. Daher muss laut DevOps auch die Infrastruktur den selben Qualitätsanspruch wie Anwendungscode entsprechen. Dies kann durch Anwendung von Qualitätssicherungs Praktiken erfolgen. \\
Darüber hinaus ist kontinuierliches Deployment mittels Automatismen wie zum Beispiel Tests oder Build Skripten ein zentraler Bestandteil von DevOps.

\section{Ziele}
DevOps verfolgt mit den bereits zuvor genannten Maßnahmen sehr wichtige Ziele. Zum einen sollen die Mitarbeiter innerhalb von Operations durch die Automatismen, die DevOps erfordert entlastet werden und zum anderen soll der Koordinationsaufwand zwischen Operations und Development stark reduziert werden. Zusätzlich soll ein vereinfachtes und weniger Fehleranfälliges Deployment von neuen Versionen einer Software ermöglicht werden.