\chapter{Zukunft von DevOps}

\section{Vorraussichtliche Marktakzeptanz}

\section{Organisatorische Probleme}

\section{Prozess Probleme}
Durch die DevOps Praktiken entstehen auch eine vielzahl an Prozess Problemen. Es unter anderem für das Continuous Deployment eine so genannte Continuous Deployment Pipeline benötigt. Eine solche Pipeline besteht aus einer Vielzahl von Werkzeugen und ist dafür gedacht, den Programmcode auf eine Plattform zu spielen. Jedes dieser Werkzeuge kann allerdings eine Abhängigkeit zu dem Hersteller des Werkzeuges zur Folge haben, was die Portierbarkeit der Anwendung erschwert. Eine Möglichkeit solche Abhängigkeiten zu verhindern besteht darin bei dem Deployment auf Standards zu setzen. Derzeit existieren allerdings noch keine akzeptierten Standards für die Werkzeuge innerhalb einer Continuous Deployment Pipeline. Durch eine weitere Verbreitung von DevOps, könnten in naher Zukunft tatsächlich Standards beschlossen werden. \\
Es existieren dennoch Strategien, um das Problem der Herstellerabhängigkeit zu minimieren. Zum einen sollte bei der Entwicklung auf einen defensiven Entwicklungsstil gepocht werden. Zum anderen existieren für eine Vielzahl von Werkzeugen Migrationswerkzeuge auf andere Werkzeuge. Diese Migrationswerkzeuge funktionieren allerdings nur bedingt gut und der Erfolg einer Migration hängt davon ab, ob nur eine Obermenge an Funktionen beider Werkzeuge verwendet werden. \\
Ein weiteres Prozess Problem, dass durch die DevOps Praktiken entsteht, ist eng mit der Häufigkeit der Deployments verbunden. Eine gut funktionierende Deployment Pipeline ermöglicht häufigere Deployments. Dies bringt eine diverse Herausforderungen mit sich. Zum einen muss die Zeit zum Testen der Anwendung verringert werden. Dies kann zur Folge haben, dass Fehler nicht rechtzeitig entdeckt werden. Daher muss ein Rollback auf eine vorherige Version bei einem Fehler möglich sein. Der Fehler könnte mit dem darauf folgenden Deployment behoben sein. Zusätzlich muss eine automatische Fehlererkennung Bestandteil der Anwendung sein, damit bei Eintreten des Fehler automatisch ein Rollback stattfinden kann. \\
Durch häufigeres Deployment sind Unternehmen in der Lage in kurzer Zeit viele neue Services anzubieten, was sowohl das Verhalten der Nutzer als auch die Last auf verschiedenen Services verändern kann. Zusätzlich kann auch das Deployment in eine Umgebung fäschlicherweise zu einem Alarm innerhalb eines Monitoring Tools der Umgebung führen. \\
All dies hat zur Folge, dass Unternehmen, die DevOps einsetzen, die Art ihres Monitorings und ihrer Qualitätssicher verändern müssen.

\section{Technologische Probleme}
In der Regel sind die Werkzeuge einer Continuous Deployment Pipeline durch Skripte zur Integration miteinander verbunden. Um weitere Werkzeuge einzufügen bedarf es einer Änderung dieser Skripte. Dieser Ansatz bringt einve Vielzahl an Problemen mit sich. Es wird durch die Skripte schwerer nachvollziehbar, welche Komponenten auf welche Plattform deployed werde. Zusätzlich enthalten Services in der Regel keine Informationen mit welchen Wekrzeugen sie deployed wurden. Daher wird eine Fehleranalyse bei einem korrupten Deployment erschwert.\\
Eine weiteres Problem entsteht dadurch, dass die Werkzeuge häufig die Sicherheitsinformationen wie Nutzername und Passwort von einem Werkzeug zum nächste weitergeben müssen und dadurch die Sicherheit gefährdet wird. \\
Das größte Problem besteht allerdings darin, dass viele Werkzeuge an sich nicht für Continuous Deployment geeignet sind, da diese in einer Zeit vor dem Continuous Deployment entstanden sind. Daher lassen sich neue Konzepte nur schwer auf diese Werkzeuge übertragen.