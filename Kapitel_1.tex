\chapter{Einleitung}
Lorem Ipsum...

\section{Developer}
Viele Betriebe wie zum Beispiel Versicherungen sind heutzutage beinahe vollständig auf Software angewiesen, die ihre Arbeit korrekt abbilden und vereinfachen. Diese Software wird von Entwicklern (Developern) mit Hilfe von Abstraktion und Programmiersprachen hergestellt. Damit die Software auch den Ansprüchen der Anwender entspricht, nimmt der Entwickler die Anforderungen vom Anwender beziehungsweise dem Auftraggeber entgegen und in Programmcode umgesetzt. Zusätzlich ist es die Aufgabe des Entwicklers den geschriebenen Programmcode sowohl manuell als auch programmatisch zu testen und damit ein korrektes Verhalten der Anwendung sicherzustellen.\\
Das Ergebnis der Arbeit eines Entwicklers ist ein Stück Software, das dem Kunden als Installationsdatei oder Ähnlichem übergeben wird. Der Entwickler selbst ist nicht dafür zuständig, dass die Anwendung in den laufenden Betrieb des Kunden integriert wird.

\section{Operation}
Unter Operations werden im herkömmlichen Sinne Mitarbeiter mit IT-Kenntnissen verstanden, deren Ziel es ist, den täglichen IT-Betrieb am laufen zu halten. Zu ihren Aufgaben gehören unter anderem die Bereitstellung von Hardware für Mitarbeiter des Betriebes sowie die Verwaltung von Servern. Sie sind ebenfalls dafür zuständig, Software unter anderem auf die Geräte der Mitarbeiter und auf die Server des Betriebes zu spielen. Dabei ist kann die Software entweder käuflich erworben oder im eigenen Unternehmen entwickelt worden sein. Dies wird in der Regel mit Skripten realisiert, die Operations selber anlegen.\\
Zu einer Software, die kommerziell erworben wurde gibt es in der Regel so gennante Service Level Agreements, die eine mit dem Softwarehersteller vereinbarte Leistungsqualität der einzusetzenden Software beschreiben. Ein Beispiel hierfür wäre, dass ein Hersteller von einer Cloudbasierten Anwendung, seinem Kunden verspricht, dass die Anwendung mit einer bestimmten Anzahl an Anfragen innerhalb einer definierten Zeitspanne zurecht kommt, sodass keine Ausfälle entstehen. Innerhalb eines Betriebes überwachen Operations die Service Level Agreements melden gegebenenfalls Verstöße ihren Vorgesetzten oder dem Hersteller der Software. \\
Um die Einhaltung der Service Level Agreements zu überprüfen, wird die Software mit Hilfe von Monitoring und Logging Werkzeugen überwacht. Dabei geben die Werkzeuge des Monitorings den Operations, wichtige Informationen wie zum Beispiel den Speicherverbrauch und den Status der laufenden Anwendungen. Mit Hilfe der Logging Werkzeuge, können die Operations auf Fehler innerhalb der Anwendung schließen. Für den Fall, dass die Umgebung der Anwendung falsch konfiguriert ist, wird dies üblicherweise in Log-Dateien geschrieben, die den Operations zur Verfügung stehen. \\
Wie bereits erwähnt, ist das Ziel von Operations den IT-Betrieb am laufen zu halten. Dies wird in der Regel als Business Continuity bezeichnet. Allerdings ist dies nicht nur mit einer für die Anwendung passend konfigurierten Umgebung möglich. Die Rechner und Server eines Betriebes müssen nach Außen abgesichert werden, um Ausfälle durch Schadangriffe so gut wie möglich zu unterbinden. Dies ist ebenfalls ein Aufgabenbereich von Operations.

\section{Agile Softwareentwicklung}

\section{Monolith}

